\documentclass{article}
\usepackage[utf8]{inputenc}
\usepackage{graphicx}
\usepackage[14pt]{extsizes}
\usepackage{amsmath, amsfonts, amssymb, amsthm, mathtools}

\title{\textbf{Taking the derivative of a function}}
\author{Timasok aka Boss of the Tree Planting Team}

\begin{document}
\maketitle
\section{Introduction}
Due to the heavy workload at the Department of Higher Cringe, the author of this article has repeatedly postponed the publication of this work. But once the heel pressed,the notebook with differentiation formulas fell out of the briefcase in front of the astonished students in the audience, into which they each burst with such enthusiasm in the hope of taking a bite out of the great knowledge of matan. After that lecture, when our editorial team saw the beaming face of Professor Dymarych Petrovich, with shaking hands turning over the abstract and whispering something, we finally decided to publish those of his drafts that will be understandable to the masses. 
\section{Work process}
\textbf{initial tree:}

$$
x^{8}
$$
\textbf{after variable replacement:}

$$
x^{8}
$$
You can check the calculations yourself
$$
8\cdot x^{(8-1)}
$$
If it seems to you that the transitions are not strict, then cross yourself
$$
8\cdot x^{7}
$$
\textbf{after partial differentiation:}

$$
8\cdot x^{7}
$$
\end{document}